\documentclass[a4paper]{article}

%% Language and font encodings
\usepackage[english]{babel}
\usepackage[utf8x]{inputenc}
\usepackage[T1]{fontenc}

%% Sets page size and margins
\usepackage[a4paper,top=3cm,bottom=2cm,left=3cm,right=3cm,marginparwidth=1.75cm]{geometry}

%% Useful Packages

\usepackage{graphicx} % To include/embedd pdf and other graphics in the doc
\usepackage{blindtext}
\usepackage{authblk}
\usepackage{amsmath}
\usepackage{ tipa }




\title{Production Analysis of Oilfield Data}


\author{Podhoretz, Seth\thanks{sbpodhoretz@gmail.com}}
\author{Kou, Rui\thanks{kourui.pete@tamu.edu}}
\author{Feng, Gan\thanks{aravikumar@tamu.edu}}
\author{Gupta, Vivek\thanks{fenggan@tamu.edu}}
\affil{Department of Statistics, Texas A \& M University}



\begin{document}
	\maketitle
\begin{abstract}
	Our project aims to utilize the abundance of oilfield data to arrive at conclusions on the physical and
	economic aspects of the flow of hydrocarbon in the reservoir. Physical aspects that we aim to analyze include
	the prevailing flow regime (the character of flow with respect to geometry and pressure drop), interference
	between wells, and communication between wells. Economic aspects include forecasting of production rates
	into the future, and estimating the ultimate hydrocarbon recovery volumes.
	
	
	The fluid that is of interest resides in pores of rock, several thousands of feet under the surface. As this
	fluid flows through the rock and into the well, the pressure in the rock drops, and the production rate drops.
	The nature of the production rate drop, and the backpressure held at the well together carry information on
	the physics of the process: the flow regime, possible interaction with another well, presence of boundaries
	in the reservoir. With these physical characteristics in mind, we can forecast the production rate into the
	future, thus calculating the economic life of the well, and the economics of production.
	
	
	Our dataset at the moment consists of 60 wells from the Bakken. The Bakken is a shale play in North
	Dakota, and one of the largest oil developments in recent history. It helped start the shale boom in the
	United States, reversing decades of declining US oil production volumes. 
\end{abstract}

\section{Introduction}
 Some Introduction
 
\section{Data Description}
 Some Introduction
\end{document}